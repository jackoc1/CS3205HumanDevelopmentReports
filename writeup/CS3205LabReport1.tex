\documentclass[11pt,a4paper,final]{article}

\usepackage[latin1]{inputenc}
\usepackage{amsmath}
\usepackage{amsfonts}
\usepackage{amssymb}
\usepackage{graphicx}
\usepackage{float}
\usepackage{fancybox}
\usepackage[T1]{fontenc}
\usepackage{babel}[English]
\usepackage[left=18mm,
            right=12mm,
            top=25mm,
            bottom=25mm]{geometry}
\usepackage[pagestyles,raggedright]{titlesec}
\usepackage{blindtext}
\usepackage{multicol}
\usepackage{color}
\usepackage{hyperref}

\hypersetup{
    colorlinks,
    linktoc=all,
    urlcolor=blue
    linkcolor=black,
}

\newcommand\fig[3]{
\begin{figure}[H]
  \centering
  \href{#3}{\includegraphics[width=\linewidth]{#1}}
  \caption{#2} 
  \label{fig:#1}
\end{figure}
}

\title{ \vspace{3.5cm}
	University College Cork \\ [1cm]
	\includegraphics[width=0.4\textwidth]{ucc_crest} \\ [1cm]
	CS3205 Lab Report 1 \\ [0.5cm]
	HDI Trends and Multidimensional Projections
}
\author{Jack O'Connor}
\date{\today}

\graphicspath{ {../hdi_visualisations/} {../multidimensional_projections} {./images/} }
\setlength{\columnsep}{0.5cm}
\pagenumbering{gobble}

\begin{document}
% Title page
\thisfancyput(3.25in,-4.5in){%
  \setlength{\unitlength}{1in}\fancyoval(7,9.5)}%
\maketitle
\pagebreak
{
\hypersetup{hidelinks}
\tableofcontents
}
\pagebreak
\setcounter{page}{1}

\begin{multicols}{2}
\pagenumbering{arabic}
\section{Introduction}
\subsection{The Problem}
the problem

\subsection{The Datasets}
the datasets

\subsection{The Role of Visualisations}
the role of visualisations


\section{Data Description}
\subsection{hdr.xlsx}
Description of data.
\subsection{hdi.csv}
Description of data.


\section{Task 1}
\subsection{Free Exploration}
In the free exploration section of the assignment I have created visualisations which give a broad overview of the general distribution of HDI across countries when looking at specific attributes of those countries.

Each of the following visualisations will include:

\begin{itemize}
	\item A graphic
	\item A description of a pattern (or lack of)
	\item A hypothesis as to the cause of this pattern
	\item The dataset from which the data came
\end{itemize}

\subsubsection{HDI Histogram}
\fig{hdi_histogram}{my hdi caption}
a

\subsubsection{Global Average HDI}
\fig{global_average_hdi_areachart}{no dips}
a

\subsubsection{Min/Max HDI Radial Plot}
\fig{min_vs_max_hdi_radialplot}{going in circles}
a

\subsubsection{Expected Population Growth}
\fig{population_growth_ratio_boxplot}{sexy visual}
a

\subsubsection{Male/Female Average Years Education}
\fig{men_vs_women_mean_years_education_barchart}{girl power}
a

\subsection{Specific Observations}
In the specific observations section of the assignment I have created visualisaions in line with the task list specifications.

\subsubsection{Spurious Values}
\subsubsection{HDI Global Heatmap}
\fig{hdi_global_heatmap}{what a pretty map}
a

\subsubsection{Correlated/Uncorrelated Attributes}
a

\subsubsection{Hypotheses for Previous Patterns}
a

\subsubsection{Usefulness of All Visualisations}
a

\subsubsection{Alternative Visualisations}
a


\section{Task 2}
Results for task 2.
\subsection{Corel Projections Comparison}
a

\subsection{CBR Projections Comparison}
a

\subsection{Dataset 1 Projections Comparison}
a

\subsection{Dataset 2 Projections Comparison}
a

\subsection{Projections of HDR}
a


\section{Conclusions}
Conclusions from overall assignment, emphasis on data sets, exercises and using visualisations.


\section{References}
References?

\end{multicols}
\end{document}